\documentclass[11pt]{article}
\author{alve}
\begin{document}
\title{the component model for expandora}
\maketitle

\section{Motivation}
While developing the new engine described in expandora.tex I noticed that 
the source code got more and more bloated and confusing. It was obviously 
impossible to separate old- and new engine specific code and in order
to compile the new engine a lot of code that wasn't used anymore in the
new engine had to be included. I partially solved that with preprocessor
directives but of course that did not solve the whole problem.
Additionally the need for a muli-user functionality where several people could 
play with a single map or multiple maps from different people could be 
synchronized at run time arose. That's why I started to split up the so far
monolithic new engine code into several distinct components.

\section{Requirements for a component}
In order to fulfill the object orientation paradigms we should rely on to
keep the code clean a component should support the general characteristics 
of an object. Among these are:
\begin{enumerate}
\item information hiding
\item inheritence from other components
\item some sort of introspection, so that external code can be informed about 
the component's capabilities at runtime
\item a means to send and receive messages
\end{enumerate}
Additionally as components should be as far as possible independend of each 
other, they should remain in seperately built entities, such as libraries.
To allow communicating components to stay on different hosts I chose a rather
simple model of serialization: A signal sent or received by a component should
only contain objects that are serializable. Objects are serializable if they 
only contain local data. Especially no serializable object should hold a pointer
to a piece of data referenced by any other object. The purpose of this 
constraint is the difficulty of referencing non-local data on remote hosts.

\subsection{Serializable Objects}
Each serializable object should subclass a abstract class named Serializable
that will provide methods for writing the information contained in this object
to a character array and reading an object from a character array. The objects
I identified to need serialization methods so far are: Room, ParseEvent, 
Frustum, Coordinate, Property and the various container classes.

\section{Implementation}
QT provides the concept of signals and slots, so it is obvious to just
make Component a subclass of QObject to fulfill requirement (4).
By choosing Component to be a subclass of QT's QObject we also get all of 
QObjects meta object features for free, thus providing feature (3). The only 
problem left will be the runtime determination of signals and slots from 
components. QObject only provides Macros SIGNAL and SLOT to translate from 
textual representations to internal representations. As we can't use these 
macros on runtime variables it is necessary to rewrite them as functions which 
will be done in the component class. Of course this is dangerous as the encoding
of signals and slots might change in future versions of QT, yet it is still
cleaner than including an own form of introspections via signal tables and slot
tables. It is also obvious that requirements (1) and (2) are also fulfilled
like this. Each component will then be in a seperate platform dependent 
library that is accessible via QLibrary. This library exports a C symbol named
createComponent that will return the new component. By using either the  meta 
objects or a predefined configurations signals and slots of various components 
can then be connected. Like this communication components can be created to
serialize and transmit or deserialize and relay signals. Of course the signals'
arguments have to be serializable to do this. To implement the actual 
serialization the QDataStream class could be helpful.

\end{document}