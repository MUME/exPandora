\documentclass[11pt]{article}
\begin{document}
\title{exPandora - An algorithm for uniquely identifying rooms in MUME}
\maketitle

\section{Motivation}
Pandora often gets confused when new rooms have similar descriptions and random exits. 
Identifying rooms in a mud like MUME is obviously an indecidable problem as every exit could be a random one with the randomness occuring only at the "wrong"
time. So we need to apply some heuristics. Additionally it is quite annoying to manually switch
between mapping and exploring perspective all the time, so a unified mode that maps rooms when we 
see them the first time is the way to go. The following algorithm tries to solve that problem as good
as possible.

\section{General approach}
Due to the limited number of relevant properties rooms and exits can have I chose to use a state
machine with some additional information in a number of stacks and qeues. Azazello already did a
great deal of work in
identifying premapped rooms, which works quite well so far. I'll use that algorithm - called
pandora algorithm from now on - as a black box to find already mapped rooms based on coordinates or
room names and as base for the new algorithm.

\section{Specification of the algorithm}
\subsection{Pandora's contribution}
In exploring mode pandora keeps a number of "specculative" rooms for the player position as soon
as it can't find a unique position anymore. These positions are eliminated with further movements
of the player until only one possible path is left. This is in itself a heuristic algorithm as it
ignores random exits. Yet it is quite a good heuristic as it most times predicts the correct
position after a short sequence of steps in premapped areas, which is proven by experience. I will
extend that behaviour to the mapping of new rooms.

\subsection{Expanding pandora}
The player can (and should) define unique rooms. These are rooms with unique properties - title,
description, exits - that only occur once in the game world. There are quite a few such rooms and the
player should be able to find them by experience and drop an "munique" when she enters one of them.
Instead of only considering premapped rooms as possible positions we also set up one experimental
path on previously unmapped rooms. So as soon as the player leaves the premapped area a new room is
created resembling the room he just enters. At the same time all existing rooms matching his new
position are found with pandora and considered possible positions, too. The experimental path is
continued - by creation of a new room and matching of existent rooms - with each step of the player
until it is either accepted or rejected. All other paths are treated as usual. The experimental path
is accepted when it meets a non-experimental unique room or when the player defines one of the rooms
he just entered as unique and there are no other paths left. Then all experimental rooms on this path
are transformed into permanent rooms. If the player defines a newly entered
but previously existing room as unique all paths that end in newly created rooms are rejected. A
path is rejected as soon as another path is accepted or if te player leaves through a known
exit and doesn't enter the expected room.

\subsection{Formal description}
We operate on rooms, exits, paths and players.
\begin{itemize}
\item{Rooms:} Each room contains pointers to up to 7 exits - n, s, e, w, u, d and "unknown". Rooms
have the property of being experimental or, on the contrary, approved. Rooms have coordinates x, y,
z, a title and a description. Two rooms are called similar if their respective titles, descriptions
and exits match up to a certain degree. A pointer to an exit is called unknown if it isn't
initialized. This is used synonymously to the exits' "unknown" property. 
\item{Exits:} A non-random, deterministic exit contains pointers to two rooms. Exits have as many
names as they have rooms. Exits can have the
property of being random. In this case they can hold an unrestricted number of room-pointers.  Exits
have the property of being known, if they connect only approved rooms. Exits "connect" the rooms they
contain. Exits are called "open" if only one of their room-pointers is not null.
\item{Paths} A path is a sequence of rooms connected with exits. A path can be enlardged to
another room which means that the new room and an exit connecting it with the previous head of the
path are added. A path can be forked to a new room which means that a new path is created from the
last exit, which is updated to contain the new room, and the new room. Weakly forking, in contrast to
normal. strongly forking, means that the fork only takes place if there are no other paths that will
strongly fork or enlarge or weakly fork and are already coordination-wise nearer to the new room
in the same step of calculation. A path
can be accepted which means that all rooms in the path get the property "approved" and all exits
connecting them get the property "known". A path can be rejected, which means that all experimental
rooms in it are deleted and all exits connnecting to one of these rooms are updated accordingly.
\item{Players:} A player contains pointers to a number of similar rooms which are called his possible
positions. A player can leave through any of the rooms' exits or in an "unknown" way and will enter a
new room determined by description, title and exits.
\end{itemize}

In general one instance of the algorithm is started for each player, so we only consider one
player. The algorithm has three states:
\begin{itemize}
\item{Approved:} We are in approved state, if the player has exactly one possible position and the
room this position refers to is approved.
\item{Experimenting:} We are in experimenting state if there are several possible positions. This
implicates that there is at least one experimental room among the possible positions.  
\item{Syncing:} In syncing state we have no clue where we are. We try to recognize our position
with each new room.
\end{itemize}


State transitions:
\subsubsection{Approved}
$=>$ There are only approved rooms and there is only one possiblel position.
\begin{itemize}
\item{The player leaves through a known exit:}
\begin{itemize}
\item{The player enters a room that matches exactly one room this exit is
connected to:} The possible position is updated to the matching room and the state remains
approved.
\item {The player enters a room matching several rooms this exit is connected to:}
Each of the matching rooms is set as a possible position and as head of a path. The state
changes to experimenting.
\item{The player enters a room that doesn't match a room this exit
is connected to:} If the exit is deterministic it is changed to random. The new room is created as
experimental and set as possible position and end of a new path. All rooms matching the new room are
found using pandora and set as possible positions and ends of a new path, too. The exit is remembered
as startOfPath and all new possible positions are added to it. The state is changed to
experimenting.
\end{itemize}
\item{The player leaves in an unknown way}
\begin{itemize}
\item{The player enters a room matching exactly one of the adjacent (by exits) rooms of her possible position:} The possible position is updated to
the matching room and the state remains approved.
\item{The player enters a room matching several of the adjacent rooms of her possible position;}
Each of the matching rooms is set as a possible position and as head of a path. The state
changes to experimenting.
\item{The player enters a room not matching any of the adjacent
rooms of her possible position:} The new room is created and set experimental and all similar rooms
are searched. All these rooms are set as heads of paths and possible positions. The unknown exit of
the old room is connected to all the these rooms. The state changes to experimenting.
\end{itemize}
\item{The player leaves through an unknown exit}
\begin{itemize}
\item{The player enters a room matching a room at the
expected coordinates:} The possible position is updated to the matching room, the exit is updated
to contain the old and the matching room and gets known. The state remains approved.
\item{Else:} The new room is created as
experimental and set as possible position and end of a new path. All rooms matching the new room are
set as possible positions and ends of a new path, too. The exit is remembered
as startOfPath and all new possible positions are added to it. The state is changed to
experimenting.
\end{itemize}
\end{itemize}
\subsubsection{Experimenting}
$=>$ There are several possible positions and experimental rooms. All newly created rooms are experimental.
\begin{itemize}
\item{The player tries to declare her current position as unique}
\begin{itemize}
\item{There is exactly one room matching the current room description and this room is a
possible position:}
The matching room is declared unique. The path this room is on is accepted. On this path all rooms
are made approved and all exits are made known. All other paths are rejected. Their experimental
rooms are deleted. The state is changed to Approved.
\item{There are several rooms matching the description but only one of them is approved:} The
approved room is declared unique. The same things as above happens with this room.
\item{Else:} Nothing happens. The player gets a message stating that she can't declare this room as
unique.
\end{itemize}
\item{One of the heads of the paths is unique:} The same happens as if the user declared this
room unique.
\item{The player leaves a room R1 (referred by its properties) through an exit E (referred by the
respecting pointer) and enters a new room R2 (referred by its properties):} For each path ending in a
possible position P:
\begin{itemize}
\item{There are rooms in P's exit E matching R2:} The path is weakly forked to each
matching room. (=> near rooms are preferred; this could be an exit between experimental rooms)
\item{P's exit E exists, it is not open, but none of its rooms matches R2:} The path is rejected. (=>
ignoring random exits)
\item{P's exit E doesn't exist yet or is open:} If there is a room matching R2 exactly at the
expected coordinates the path is enlard by this room. Else one new room matching R2 is created, the
path is enlarged by this room and the path is weakly forked to each other room matching R2. (=> we
prefer coordination-wise correct links to "incorrect" links.)
\end{itemize}


\item{The player leaves in an unknown way:} For performance reasons we reject all paths and enter
the Syncing state.
\end{itemize}
If there is only one path left this path is accepted and the Approved state is entered. If there
is no path left Syncing state is entered.
\subsubsection{Syncing}
In this state the pandora algorithm is used to find the position. We enter the Approved state as
soon as we know for sure where we are.


\subsection{Quality of the algorithm}
As we only do weak forking in the Experimenting state it is guaranteed that all paths end in different rooms at all times 
and we enter Approved state as soon as we meet a unique room. To tell how correct the resulting path is we would need 
to know the properties of MUME's real map which isn't the case. What we can tell is that most of MUME's map can be
expressed in a three-dimensional coordinate-system with exits between adjacent rooms. As long as we follow this system
the mapping works well as adjacent matching rooms are always preferred over far range exits. Still this could yield several 
paths spawning off when many tuples of similar rooms are close to each other. But as it is impossible to get to a unique room 
on two seperate paths the user can end this behaviour by finding an approved unique room. If the map strictly conforms to the
coordinate system, there will be one path that enters the unique room from a directly adjacent room and this is the one that will be approved.
This could either be the simple following of coordinates which is always tried or a path that "jumps" around and could also be tried. 
As adjacent connections are preferred over jumps that second path can never get a room the first one takes. So the first one will be approved.
As I don't know much about the properties of far range links I can only state that their mapping must be tested by experiment.

\end{document}